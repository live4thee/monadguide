\section{动机}

Haskell是一门纯函数式编程语言。用Haskell编写的函数是引用透明(referentially transparent)的。直觉上来说,它的意思是,对于相同的给定参数,函数总是返回同样的结果。更正式一点讲,给定一函数$f$,把对$f$的调用替换成调用结果,对程序的意义没有任何影响。所以,假如$f\ 3 = foo$,你可以安全的把所有出现$f\ 3$的地方都替换为$foo$,反之亦然。“纯函数式”意味着该语言不允许副作用(side-effects),因为它会破坏引用透明。这样一来,函数的结果仅仅依赖于给定参数,也就是说,没有副作用。

Haskell中的函数和数学意义上的函数很相似,这使得对代码进行推理(reasoning)更加简单,而且相比于非引用透明的代码,编译器很多情况下可以做出更好的优化。进一步讲,求值顺序变得毫无意义。对于表达式$(x,y)$,编译器可以自由选择先对$x$还是$y$进行求值。如果其中一个值并不需要,甚至可以忽略对其求值。这带来了灵活性(如果仅仅是使用其有限部分,你可以拥有无穷的数据结构或者计算\footnote{译者注:比如Scheme中的Lazy List})和高性能。最后,编译器还可以选择获得结果的任意执行路径,从而导致Haskell程序具有近乎疯狂的可并行性,因为编译器可以同时并行执行多条路径。比如,它可以决定同时计算出$x$和$y$。

与引用透明相反的是引用模糊(referentially opaque)。一个引用模糊的函数有多重含义,甚至对于相同参数会返回不同结果,而标准示例便是随机数生成器。在大多数程序语言中,随机数函数根本就不需要任何参数。对于一个仅仅向屏幕打印出固定文本,并且总是返回$0$的函数来讲,虽然有点违反直觉,它也是个引用模糊的函数。因为你无法在不改变程序意义的条件下,把对该函数的调用全部替换成$0$。

正如上文指出,一个明显的结果是Haskell中无法写出一个不带参数的返回伪随机数的$random$函数,因为这会破坏引用透明。事实上,Haskell中一个不带参数的函数根本就不是函数,它仅仅是个值。这个问题存在很多简单的解决方案,其中一个方法是在输入参数中引入一个状态值,函数返回伪随机数的同时也返回一个新的状态。

\begin{lstlisting}
random :: RandomState -> (Int, RandomState)
\end{lstlisting}

另一个方法是用一个参数作为初始种子值,然后返回一个包含伪随机数的无穷表。利用上面定义的$random$,该函数可以简单的实现为:

\begin{lstlisting}
randomList :: RandomState -> [Int]
randomList state = x : randomList newState
  where
    (x, newState) = random state
\end{lstlisting}

由上述示例我们得出,确定性序列的的问题可以轻松搞定,而且相比于命令式语言中常见的引用模糊的$random$函数,我们得到一个有用的特性:状态可以序列号,并且简单地回溯到先前的状态,或者向两个函数注入相同的伪随机数序列。

如何处理输入输出呢?一门通用语言如果不能开发用户接口或者读文件几乎毫无用处。我们终究想要从键盘读取输入或者向终端打印一些东西。假想遇到一个$getChar$函数,它从终端读取单个字符:

\begin{lstlisting}
getChar :: Char
\end{lstlisting}

你会发现该函数违背了引用透明,因为每次调用该函数都可能返回一个不同的字符。前面已经看到,该问题可以通过引入一个状态变量解决。但我们需要什么状态?终端的状态吗?嗯,让我们使之更通用一点,传入宇宙的状态,假设它类型是$Universe$。这样我们可以修改$getChar$函数的类型,并且实现一个$twoChars$函数以展现如何使用$getChar$:

\begin{lstlisting}
getChar :: Universe -> (Char, Universe)

twoChars :: Universe -> (Char, Char, Universe)
twoChars world0 = (c1, c2, world2)
  where
    (c1, world1) = getChar world0
    (c2, world2) = getChar world1
\end{lstlisting}

我们似乎已经找到一种有效的解决方案来应付这种问题 -- 只要传递一个状态变量即可。但该方法有个问题。首先,当然,因为需要额外传递状态,程序员需要付出更多的击键;其次,更重要的是,像随机函数$random$这样的有用而必要的功能,却成为读取键盘和写终端之类严格非纯(strictly impure)操作的主要障碍:

\begin{lstlisting}
strangeChars :: Universe -> (Char, Char)
strangeChars world = (c1, c2)
  where
    (c1, _) = getChar world
    (c2, _) = getChar world
\end{lstlisting}

让我们试着理解以上代码的意图究竟是什么。我们从宇宙中,也就是世界的状态, 读取字符$c1$,同样的状态里还读取了字符$c2$,于是我们其实时光旅行到了过去。但这是何时发生的呢?首先,计算$c1$和$c2$的顺序是未定义的,因为我们没有像在$twoChars$函数里那样序列化世界的状态。 其次,$strangeChars$没有返回更新后的宇宙状态,当数据读出之后,这事儿被忘记得一干二净\footnote{译者注:因为世界的状态没有更新},就像压根儿没发生过一样。

结论:我们可以无痛地序列化伪随机数生成器的状态,对于宇宙的状态却无能为力\footnote{译者注:伪随机数生成器可以通过公式(如线性同余算法)由上一个数值计算出下一个值,而对于宇宙状态来说,是没有公式可以描述的。}。该问题存在一些解决方案。比如,纯函数式语言$Clean$,它使用了一个$uniqueness$类型,基本上就像上文描述的$universe$那样。但是,该语言会检测并阻止任何对世界状态的并行访问的尝试,从而保证了其基于显式状态传递的I/O一致性。Haskell则采用了另一种方案。它引入了一种范畴论中称之为单子的结构,而不是显式地传递世界的状态。