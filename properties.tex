\section{Monad的属性}
\subsection{Monad定律}
首先,要成为Monad,它必须遵守三个法则。除了范畴论的需要外,这三个法则也确实有意义:
\begin{enumerate}
\item $return\ x >>=  f \equiv f\ x$
\item $c >>= return \equiv c$
\item $c >>= (\backslash{}x \rightarrow f\ x >>= g) \equiv (c >>= f) >>= g$
\end{enumerate}

从绑定的角度来讲,第一条法则要求$return$是个左幺元。简单来说,它的意思是,$return$将$x$变成一个求值结果是$x$的计算,然后绑定至消费者函数$f$,这个过程等同于直接用$x$调用$f$。听起来很明显,不是吗?

从绑定的角度来讲,第二条法则要求$return$是个右幺元。意思是,将一个源计算的结果$x$绑定至$return$函数(它会给出一个返回结果$x$的计算)等同于直接给出源计算(求值为$x$)。这应该和上一条法则一样显而易见。

第三条法则要求绑定子函数$>>=$满足结合律。要明白该法则,我们先考虑一个返回给定整数三十次方根的计算(如果没有结果则返回$Nothing$)。你可以先求平方根,再依次求立方根和五次方根:

\begin{lstlisting}
i30throot :: Integer -> Maybe Integer
i30throot x = isqrt x >>= icbrt >>= i5throot
\end{lstlisting}

你可以首先得到求六次方根的计算$(isqrt\ x >>= icbrt)$,然后将其结果传递给$i5throot$求五次方根;或者你可以先求平方根$(isqrt x)$,然后将其结果传递给求十五次方根的计算$(\backslash{}y \rightarrow icbrt\ y >>= i5throot)$。第三条法则要求这两种计算的最终结果是相同的。

\subsection{辅助函数}
前面讲过,上面给出的$Monad$类并不完整。它还包含两个成员函数,$(>>)$和$fail$:

\begin{lstlisting}
class Monad m where
  return :: a -> m a
  (>>=)  :: m a -> (a -> m b) -> m b
  (>>)   :: m a -> m b -> m b
  fail   :: String -> m a
\end{lstlisting}

假如一个计算的结果没有意义,或者并不需要,用$(>>)$函数是非常方便的。你不必在实例化该函数时提供它的定义,因为它可以简单从$(>>=)$函数推导而来,下面便是其默认定义:

\begin{lstlisting}
a >> b = a >>= const b
\end{lstlisting}

假设有$a$和$b$两个计算,然后你想得到一个计算,它是运行$a$和$b$后的结果,该计算的求值结果便是$b$的结果,而忽略$a$。通过绑定子函数,你可以撰写代码如$a >>= \backslash\_ \rightarrow b$,或者$ a >>= const b$。有了$(>>)$函数后我们可以方便的写为$a >> b$。本节后面将会讲到隐式状态,此时$(>>)$函数将变得很有用。

该函数常常用在求值为类型$()$的计算中。单元类型$()$常用于求值结果没有意义的情况。和$\bot$不同的是,$()$类型只有一个值,也就是$()$。该单元类型类似于C语言中的$void$类型,只是C的$void$类型没有取值。

$fail$函数接受一个出错字符串作为参数,返回一个代表失败的计算,而且该计算中很有可能包含给定出错字符串。$fail$函数的默认定义是返回一个$\bot$计算,因此调用该函数将会终止程序。后面会讲到该函数的用处,我们现在可以先忽略它。
