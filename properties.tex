\section{Monad的属性}
\subsection{Monad定律}
首先,要成为Monad,它必须遵守三个法则。除了范畴论的需要外,这三个法则也确实有意义:
\begin{enumerate}
\item $return\ x >>=  f \equiv f\ x$
\item $c >>= return \equiv c$
\item $c >>= (\backslash{}x \rightarrow f\ x >>= g) \equiv (c >>= f) >>= g$
\end{enumerate}

从绑定的角度来讲,第一条法则要求$return$是个左幺元。简单来说,它的意思是,$return$将$x$变成一个求值结果是$x$的计算,然后绑定至消费者函数$f$,这个过程等同于直接用$x$调用$f$。听起来很明显,不是吗?

从绑定的角度来讲,第二条法则要求$return$是个右幺元。意思是,将一个源计算的结果$x$绑定至$return$函数(它会给出一个返回结果$x$的计算)等同于直接给出源计算(求值为$x$)。这应该和上一条法则一样显而易见。

第三条法则要求绑定子函数$>>=$满足结合律。要明白该法则,我们先考虑一个返回给定整数三十次方根的计算(如果没有结果则返回$Nothing$)。你可以先求平方根,再依次求立方根和五次方根:

\begin{lstlisting}
i30throot :: Integer -> Maybe Integer
i30throot x = isqrt x >>= icbrt >>= i5throot
\end{lstlisting}

你可以首先得到求六次方根的计算$(isqrt\ x >>= icbrt)$,然后将其结果传递给$i5throot$求五次方根;或者你可以先求平方根$(isqrt x)$,然后将其结果传递给求十五次方根的计算$(\backslash{}y \rightarrow icbrt\ y >>= i5throot)$。第三条法则要求这两种计算的最终结果是相同的。
