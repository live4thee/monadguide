% !TEX TS-program = xelatex
% !TEX encoding = UTF-8

% This is a simple template for a XeLaTeX document using the "article" class,
% with the fontspec package to easily select fonts.

\documentclass[11pt]{article} % use larger type; default would be 10pt

\usepackage{fontspec} % Font selection for XeLaTeX; see fontspec.pdf for documentation
\defaultfontfeatures{Mapping=tex-text} % to support TeX conventions like ``---''
\usepackage{xunicode} % Unicode support for LaTeX character names (accents, European chars, etc)
\usepackage{xltxtra} % Extra customizations for XeLaTeX

\usepackage[colorlinks=true,pdfstartview=FitH,bookmarks=true]{hyperref}

% other LaTeX packages.....
\usepackage{geometry} % See geometry.pdf to learn the layout options. There are lots.
\geometry{a4paper} % or letterpaper (US) or a5paper or....
%\usepackage[parfill]{parskip} % Activate to begin paragraphs with an empty line rather than an indent

\usepackage{listings}
\lstset{frame=leftline,basicstyle=\sffamily,language=Haskell}
\usepackage{zhspacing}

\title{Undertanding Haskell Monads}

\author{Author: Ertugrul S\"oylemez\\
\url{http://ertes.de/articles/monads.html}\\[2mm]
翻译:李群}

\date{\today\\
Version 1.03}

\zhspacing
\begin{document}
\maketitle

Haskell 是一门流行的现代函数式程序设计语言,它易于学习,语法优美,并且相当富有生产力。可是学习Haskell的过程中有一个最大的障碍,那便是Monad。虽然它们其实相当简单,但初学者很容易对其感到困惑。 我对于将Haskell推广到现实应用深怀兴趣,因此写下这篇介绍。

\clearpage
\renewcommand\contentsname{目录}
\tableofcontents
\clearpage

\section{序言}

这篇教程是写给对Haskell语言已经有了基本了解的新手的,你们或许曾经尝试理解Haskell的一个关键概念,也就是单子。假如你理解起来有难度,或者确实已经理解,但还是想有更深入的洞察或者了解其背后的动机或背景,本文便是为你而作。

Haskell是一门函数式语言。这一点并无特殊之处,但其设计使它易于学习、理解,并且实践中非常有效(effect)和高效(efficient)。Haskell有一个非常特殊的特性,也就是泛化(generalization)的概念。它的意思是,相比于直接实现一个想法,你更乐意试着寻找一个更通用的概念,使得前者是其一个隐含的特例。这种方法的优点是,假若将来又找到另一个特例,你不必重新实现之,至少不必全部推到重来。

传统的程序员从未面对过泛化问题,最多只是面向对象中的抽象概念而已。程序员更喜欢具体、特化的概念,就像直白如“干活的工具“。不幸的是,这种态度仍然普遍存在。单子的概念就是一个尤其悲情的例子,虽然单子极其有用,但是Hakell新手往往容易放弃深入理解它,因为单子是个非常抽象的结构,使得在其基础之上实现的功能也处于一个难以置信的通用级别。

或许某些读者读过 Brent Yorgey 的 《抽象、直觉以及”单子教程谬误“》,书中清楚地阐明,向Haskell新手提供单子的又一种解释为何总是于事无补。环顾所有,我将上述问题视为一个额外的麻烦:虽然掌握了抽象概念后会带来诸多便利,但很多人都试图抵制它们。因此,这份教程中我的主要意图便是帮助读者击败对抽象概念的恐惧,并阐明单子的本质:它不过是组合计算的一种风格的抽象而已。

我希望这篇文章能够给予你帮助,若有任何建设性的反馈,我将非常感谢。\clearpage
\section{Monad之由来}

Haskell是一门纯函数式编程语言。用Haskell编写的函数是引用透明(referentially transparent)的。直觉上来说,它的意思是,对于相同的给定参数,函数总是返回同样的结果。更正式一点讲,给定一函数$f$,把对$f$的调用替换成调用结果,对程序的意义没有任何影响。所以,假如$f\ 3 = foo$,你可以安全的把所有出现$f\ 3$的地方都替换为$foo$,反之亦然。“纯函数式”意味着该语言不允许副作用(side-effects),因为它会破坏引用透明。这样一来,函数的结果仅仅依赖于给定参数,也就是说,没有副作用。

Haskell中的函数和数学意义上的函数很相似,这使得对代码进行推理(reasoning)更加简单,而且相比于非引用透明的代码,编译器很多情况下可以做出更好的优化。进一步讲,求值顺序变得毫无意义。对于表达式$(x,y)$,编译器可以自由选择先对$x$还是$y$进行求值。如果其中一个值并不需要,甚至可以忽略对其求值。这带来了灵活性(如果仅仅是使用其有限部分,你可以拥有无穷的数据结构或者计算\footnote{译者注:比如Scheme中的Lazy List})和高性能。最后,编译器还可以选择获得结果的任意执行路径,从而导致Haskell程序具有近乎疯狂的可并行性,因为编译器可以同时并行执行多条路径。比如,它可以决定同时计算出$x$和$y$。

与引用透明相反的是引用模糊(referentially opaque)。一个引用模糊的函数有多重含义,甚至对于相同参数会返回不同结果,而标准示例便是随机数生成器。在大多数程序语言中,随机数函数根本就不需要任何参数。对于一个仅仅向屏幕打印出固定文本,并且总是返回$0$的函数来讲,虽然有点违反直觉,它也是个引用模糊的函数。因为你无法在不改变程序意义的条件下,把对该函数的调用全部替换成$0$。

正如上文指出,一个明显的结果是Haskell中无法写出一个不带参数的返回伪随机数的$random$函数,因为这会破坏引用透明。事实上,Haskell中一个不带参数的函数根本就不是函数,它仅仅是个值。这个问题存在很多简单的解决方案,其中一个方法是在输入参数中引入一个状态值,函数返回伪随机数的同时也返回一个新的状态。

\begin{lstlisting}
random :: RandomState -> (Int, RandomState)
\end{lstlisting}

另一个方法是用一个参数作为初始种子值,然后返回一个包含伪随机数的无穷表。利用上面定义的$random$,该函数可以简单的实现为:

\begin{lstlisting}
randomList :: RandomState -> [Int]
randomList state = x : randomList newState
  where
    (x, newState) = random state
\end{lstlisting}

由上述示例我们得出,确定性序列的的问题可以轻松搞定,而且相比于命令式语言中常见的引用模糊的$random$函数,我们得到一个有用的特性:状态可以序列号,并且简单地回溯到先前的状态,或者向两个函数注入相同的伪随机数序列。

如何处理输入输出呢?一门通用语言如果不能开发用户接口或者读文件几乎毫无用处。我们终究想要从键盘读取输入或者向终端打印一些东西。假想遇到一个$getChar$函数,它从终端读取单个字符:

\begin{lstlisting}
getChar :: Char
\end{lstlisting}

你会发现该函数违背了引用透明,因为每次调用该函数都可能返回一个不同的字符。前面已经看到,该问题可以通过引入一个状态变量解决。但我们需要什么状态?终端的状态吗?嗯,让我们使之更通用一点,传入宇宙的状态,假设它类型是$Universe$。这样我们可以修改$getChar$函数的类型,并且实现一个$twoChars$函数以展现如何使用$getChar$:

\begin{lstlisting}
getChar :: Universe -> (Char, Universe)

twoChars :: Universe -> (Char, Char, Universe)
twoChars world0 = (c1, c2, world2)
  where
    (c1, world1) = getChar world0
    (c2, world2) = getChar world1
\end{lstlisting}

我们似乎已经找到一种有效的解决方案来应付这种问题 -- 只要传递一个状态变量即可。但该方法有个问题。首先,当然,因为需要额外传递状态,程序员需要付出更多的击键;其次,更重要的是,像随机函数$random$这样的有用而必要的功能,却成为读取键盘和写终端之类严格非纯(strictly impure)操作的主要障碍:

\begin{lstlisting}
strangeChars :: Universe -> (Char, Char)
strangeChars world = (c1, c2)
  where
    (c1, _) = getChar world
    (c2, _) = getChar world
\end{lstlisting}

让我们试着理解以上代码的意图究竟是什么。我们从宇宙中,也就是世界的状态, 读取字符$c1$,同样的状态里还读取了字符$c2$,于是我们其实时光旅行到了过去。但这是何时发生的呢?首先,计算$c1$和$c2$的顺序是未定义的,因为我们没有像在$twoChars$函数里那样序列化世界的状态。 其次,$strangeChars$没有返回更新后的宇宙状态,当数据读出之后,这事儿被忘记得一干二净\footnote{译者注:因为世界的状态没有更新},就像压根儿没发生过一样。

结论:我们可以无痛地序列化伪随机数生成器的状态,对于宇宙的状态却无能为力\footnote{译者注:伪随机数生成器可以通过公式(如线性同余算法)由上一个数值计算出下一个值,而对于宇宙状态来说,是没有公式可以描述的。}。该问题存在一些解决方案。比如,纯函数式语言$Clean$,它使用了一个$uniqueness$类型,基本上就像上文描述的$universe$那样。但是,该语言会检测并阻止任何对世界状态的并行访问的尝试,从而保证了其基于显式状态传递的I/O一致性。Haskell则采用了另一种方案。它引入了一种范畴论中称之为``Monad''的结构,而不是显式地传递世界的状态。\clearpage
\section{一则示例}

阅读这份教程时,或许你已经知道单子可能有多种解释。它们是个抽象结构,而第一个难点是它们到底可以用在哪。单子主要有两种解释:容器(container)和计算(computation)。这很好的解释了现有单子的用途,但就作者个人浅见,读者还是无法从中得知怎样辨识单子,而且这两种解释偶尔看起来不太兼容(虽然并非如此)。

So I'm trying to provide you with an idea of monads, that is more generic and allows you to find familiar patterns in monad usage. Particularly it should make it easy to recognize monads as such. However, I won't give you a concrete notion yet. Let's start with a motivating example instead.

比方说,你有个函数,但是它未必能返回一个结果,那么如何定义的它的类型呢?对整数求精确平方根是个好的例子,我们可以书写如下代码:

\begin{lstlisting}
isqrt :: Integer -> Integer
\end{lstlisting}

好了,那么$isqrt\ 3$的结果是什么?$isqrt\ (-5)$呢?我们如何处理计算没有结果的情况呢?第一个主意来自命令式编程世界,我们期望传入参数是个$Square$类型,否则便通过异常或者信号来终止程序的运行。我们说,假如参数不是一个$Square$值,则函数结果是个$\bot$或者$bottom$。

$\bot$值是个理论概念。它可以作为函数的返回值,但对它求值将永远不会返回,因此你永远无法直接观察它的确切值。无穷递归函数和抛出异常的函数便是这方面的例子。两者都没有返回一个普通函数值。

数学上来讲,如果非法参数根本就无法传入,这样的手段显然更为高明:

\begin{lstlisting}
isqrt :: Square -> Integer
\end{lstlisting}

第一眼看上去它的确能工作,但是假如我们需要计算一个整型值的平方根呢?这里我们需要首先将其转换成$Square$类型,于是我们又回到了老路上\footnote{译者注:如果此时给定一个负整数,转成$Square$类型后再计算会得到正数平方根。因此这里从负数变成$Square$是个无意义的计算。}。而且很多时候我们确实希望程序能处理这种没有结果的情况。于是我们可以写个包装类型$Maybe$:

\begin{lstlisting}
data Maybe a = Just a | Nothing
\end{lstlisting}

类型为$Maybe\ a$的值只可能二者取一:要么是$Nothing$,要么是$Just\ x$。这里$x$的类型是$a$。 比如,类型为$Maybe\ Integer$的值要么是$Nothing$,要么是$Just\ x$,而$x$的类型是$Integer$。现在我们修改一下整数平方根函数的类型,捎带其实现(虽然不是最理想的,但很容易看懂)如下:

\begin{lstlisting}
isqrt :: Integer -> Maybe Integer
isqrt x = isqrt' x (0,0)
  where
    isqrt' x (s,r)
      | s > x     = Nothing
      | s == x    = Just r
      | otherwise = isqrt' x (s + 2*r + 1, r+1)
\end{lstlisting}

由于$Nothing$是个有效返回值,我们的函数必须处理所有情况。下面是一些$isqrt$返回值的示例:

\begin{lstlisting}
isqrt 4  = Just 2
isqrt 49 = Just 7
isqrt 3  = Nothing
\end{lstlisting}

假如我们想求四次方根呢?既然已经有了个平方根函数,在此之上写个四次方根函数岂不是更酷? 如果没有计算结果,我们可以照样返回$Nothing$。这个函数该怎么写呢?我很有信心您很快就会写出如下代码:

\begin{lstlisting}
i4throot :: Integer -> Maybe Integer
i4throot x = case isqrt x of
               Nothing -> Nothing
               Just y  -> isqrt y
\end{lstlisting}

请尽可能的理解上述代码。它首先对传入参数求平方根。如果无解,自然四次方根也无解;如果有解,则再对其求一次平方根 -- 当然求值可能再次失败, 所以这是个$Maybe$计算,由两次$Maybe$计算所得。

您理会其他包装类型,如列表,和它的相似性了吗?你可以用列表实现如下代码:

\begin{lstlisting}
isqrt :: Integer -> [Integer]
isqrt = ...

i4throot :: Integer -> [Integer]
i4throot x = case isqrt x of
               []  -> []
               [x] -> isqrt x
\end{lstlisting}

当然,列表也允许多个求值结果,于是您甚至可以在$isqrt$中同时返回给定数值的两个平方根。你可以在$i4throot$中同样考虑这个因素,虽然现在$isqrt$只是返回一个平方根。

The general idea is: You have some computation with a certain type of result and a certain structure in its result (like allowing no result, or allowing arbitrarily many results), and you want to pass that computation's result to another computation. Wouldn't it be great to have a nice generic syntax for this idea of combining computations, which are built from smaller computations, and which use such a wrapper type like Maybe or lists? It would be even better to abstract away certain structural properties of the result, so the fact that lists allow multiple values would be taken into account implicitly. We're approaching monads. 
\clearpage
\section{单子}
上一节您已经了解了一些单子的基本概念。单子是一个类型(内部类型)的包装类型,使得内部类型具有某种结构,从而您可以通过某种方式对内部类型进行组合运算。这其实就是单子的全部奥义。

为了做到这一点,您只能间接访问单子中包装的内部类型。比方说,$Maybe$类型其实就是个单子。它为内部类型添加了一个结构,可能没有返回值,或者返回一个可以求得$3$的计算,亦即$Just\ 3$,而不是直接返回$3$。继而,你可能得到一个计算,它根本不会求得某个值,也就是$Nothing$。

现在你可能希望用某个计算的结果去创建另一个计算,这是单子最重要的特性之一 -- 它允许您从简单的计算单元构造出复杂计算,并且中间不去要对这些计算求值。

边注:为了强调单子并无特殊之处,也不是什么怪异魔法,我暂时先避免使用Haskell那些讨人喜欢的语法糖,我将自己实现一些单子,简单直白的使用它们。

技术上来讲,如果您发现你的某个数据类型是单子,你可以将它实现成$Monad$类型类(type class)的一个实例。下面是$Monad$类的部分重要代码(该类还有两个成员函数,我们暂时还不需要它们):

\begin{lstlisting}
class Monad m where
  return :: a -> m a
  (>>=)  :: m a -> (a -> m b) -> m b
\end{lstlisting}


$return$函数,正如它的类型暗示的那样,它接受一个内部类型的值,给出一可以运算得出该值的计算。$(>>=)$函数(通常用其中缀记法$>>=$)更有趣: $c >>= f$是个计算,它把计算$c$的结果丢给$f$。你可以把该成员函数看做用于将中间值($c$的运算结果)传入到更大的计算($f$的运算结果)。我们很快便会看清楚它究竟看起来是啥样。从现在开始,我将$c$称为源计算,$f$称为消费者函数。

$(>>=)$函数常被称为“绑定子(bingding operator)”或者“绑定函数(bind function)”,因为它将一个计算结果绑定到另一个更大的计算。函数$f$,也就是$(>>=)$的第二个参数,我将在下文中称之为单子函数(monadic function)。该函数返回一个计算。

函数$(>>=)$的威力在于,传给消费者函数的源计算的结果完全是未指明的。在$Maybe$单子中,它可能不会有返回值,这种情况下消费者函数永远不会被调用到。

\subsection{Maybe 单子}

从一个实例出发来理解会更简单一点。暂时想象$Maybe$单子并不存在,让我们来定义它:

\begin{lstlisting}
data Maybe a = Just a | Nothing   deriving (Eq, Show)
\end{lstlisting}

你会很快发现,$Maybe$类型其实是一个内部类型的包装。$Maybe$类型的$return$函数很容易实现:在$Maybe$单子中, $Just\ x$是个计算,对它求值会得到$x$。

更有趣的是,比方说,你需要在某个计算中使用一个$Maybe$计算(源计算)的结果,将它绑定到变量$x$,假如源计算含有一个结果,那该结果就是$x$,假如没有结果,则整个计算没有结果。该想法是通过绑定子函数$(>>=)$实现的。

\begin{lstlisting}
instance Monad Maybe where
  return x = Just x

  Nothing >>= f = Nothing
  Just x  >>= f = f x
\end{lstlisting}

将计算$Nothing$传递给消费者函数$f$以为着立即返回$Nothing$(因为无结果可用)而无需调用$f$;若是传递进来的是$Just\ x$,则意味着将$x$传递给$f$并返回该结果。

现在让我们重新看看整型值精确四次方根函数$i4throot$。它首先计算传入参数的精确平方根,假如有值,则对其再算一次精确平方根。这听起来是不是很熟悉?当然应该如此,现在我们可以开发利用一下$Maybe$的单子属性(你绝对应该试一试):

\begin{lstlisting}
i4throot :: Integer -> Maybe Integer
i4throot x = isqrt x >>= isqrt
\end{lstlisting}

\subsection{列表 `[]' 单子}

和$Maybe$类型类似,列表类型也是个包装类型。它为内部类型添加了一些逻辑结构,可以返回多个结果。将其计算结果传递给一个单子函数会发生什么情况?

要回答这个问题,先问问自己:首先,什么是列表?它代表任意多个值,因而也代表了非确定性(non-determinism)。换言之,计算$[2,3,7]$仅仅是一个,它的运算结果是$2$, $3$ 和 $7$。

好了,现在想象一下你在某个计算中需要用到列表计算的结果,于是你想把列表计算的结果绑定到变量$x$。假如$x$中没有结果,则整个计算也没有结果,就像$Maybe$一样。假如列表中有一个结果,则它被绑定到$x$。假如列表中有两个结果呢?如果$x$对应多个值,则消费者函数也会魔术般被调用多次,每次传入$x$绑定的一个值。最终结果则是所有这些单个结果的集合。

\begin{lstlisting}
[10,20,40] >>= \x -> [x+1, x+2]
\end{lstlisting}

求值这个计算将会得到结果$[11,12,21,22,41,42]$,因为你把计算$[10,20,40]$的结果绑定到$x$,其中$x$包含三个结果,于是计算$[x+1, x+2]$将会执行三次,$x$的值分别是$10$,$20$和$40$。三次计算结果将会被合并到最终一个大的运算结果。

$return$函数在这里还是很容易实现,在列表单子中,$[x]$是个求值为$x$的计算。看一看如何实例化列表类型的单子:

\begin{lstlisting}
instance Monad [] where
  return x = [x]
  xs >>= f = concat . map f $ xs
\end{lstlisting}

$(>>=)$函数还是很有趣:它首先把消费者函数$f$映射到源列表中的所有值,也就是用$xs$中的所有值调用了一遍$f$。因为$f$的每次调用本身都会返回一个列表,映射得到的结果是个列表的列表。第二步,它连结列表中的所有成员列表形成一个大列表。下面的代码再次阐明了这一切。

\begin{lstlisting}
multiples :: Num a => a -> [a]
multiples x = [x, 2*x, 3*x]

testMultiples :: Num a => [a]
testMultiples = [2,7,23] >>= multiples
\end{lstlisting}

$testMultiples$ 的结果是 $[2,4,6,7,14,21,23,46,69]$。它首先从源列表中取各个值,然后将它们传入消费者函数$multiples$,得到三个列表作为结果:$[2,4,6]$, $[7,14,21]$ 和 $[23,46,69]$。换句话说,它将函数$multiples$映射于源列表。最终,连结三个列表得到一个包含所有结果的列表。

\subsection{Identity单子}

$Identity$单子实际中比较少用,为了接下来描述$State$和$IO$单子,这里先稍微介绍一下。当然,这也是为了单子理论的完整性。你也可能在单子转换器(Monad Transformer)中看到对$Identity$单子的使用,这些我稍后再讲。

$Identity$单子包装的计算只是简单的返回一个值而已,并不像$Maybe$或者列表单子那样假如额外逻辑结构。将一个$identity$计算的结果绑定到消费者函数仅仅意味着原样传入源计算的结果。

\begin{lstlisting}
data Identity a = Identity a

instance Monad Identity where
  return = Identity
  Identity x >>= f = f x
\end{lstlisting}

$Identity 5$是一个计算,其值为$5$。因为$return\ x$需要返回一个求值为$x$的计算,所以很自然,$return\ x$只需直接简单地返回$Identity x$。将$x$传递给计算$Identity\ x$后再传给消费者函数$f$也仅仅意味着直接将结果$x$传递给$f$。
\clearpage
\section{Monad的属性}
\subsection{Monad定律}
首先,要成为Monad,它必须遵守三个法则。除了范畴论的需要外,这三个法则也确实有意义:
\begin{enumerate}
\item $return\ x >>=  f \equiv f\ x$
\item $c >>= return \equiv c$
\item $c >>= (\backslash{}x \rightarrow f\ x >>= g) \equiv (c >>= f) >>= g$
\end{enumerate}

从绑定的角度来讲,第一条法则要求$return$是个左幺元。简单来说,它的意思是,$return$将$x$变成一个求值结果是
$x$的计算,然后绑定至消费者函数$f$,这个过程等同于直接用$x$调用$f$。听起来很明显,不是吗?

从绑定的角度来讲,第二条法则要求$return$是个右幺元。意思是,将一个源计算的结果$x$绑定至$return$函数(它会
给出一个返回结果$x$的计算)等同于直接给出源计算(求值为$x$)。这应该和上一条法则一样显而易见。

第三条法则要求绑定子函数$>>=$满足结合律。要明白该法则,我们先考虑一个返回给定整数三十次方根的计算(如果
没有结果则返回$Nothing$)。你可以先求平方根,再依次求立方根和五次方根:

\begin{lstlisting}
i30throot :: Integer -> Maybe Integer
i30throot x = isqrt x >>= icbrt >>= i5throot
\end{lstlisting}

你可以首先得到求六次方根的计算$(isqrt\ x >>= icbrt)$,然后将其结果传递给$i5throot$求五次方根;或者你可以先求
平方根$(isqrt x)$,然后将其结果传递给求十五次方根的计算$(\backslash{}y \rightarrow icbrt\ y >>= i5throot)$。第
三条法则要求这两种计算的最终结果是相同的。

\subsection{辅助函数}
前面讲过,上面给出的$Monad$类并不完整。它还包含两个成员函数,$(>>)$和$fail$:

\begin{lstlisting}
class Monad m where
  return :: a -> m a
  (>>=)  :: m a -> (a -> m b) -> m b
  (>>)   :: m a -> m b -> m b
  fail   :: String -> m a
\end{lstlisting}

假如一个计算的结果没有意义,或者并不需要,用$(>>)$函数是非常方便的。你不必在实例化该函数时提供它的定义,
因为它可以简单从$(>>=)$函数推导而来,下面便是其默认定义:

\begin{lstlisting}
a >> b = a >>= const b
\end{lstlisting}

假设有$a$和$b$两个计算,并且你想获得一个计算,它是运行$a$和$b$后的结果,且该计算的求值结果便是$b$的结
果,而忽略$a$。通过绑定子函数,你可以撰写代码如$a >>= \backslash\_ \rightarrow b$,或者$ a >>= const b$。
有了$(>>)$函数后我们可以方便的写为$a >> b$。本节后面将会讲到隐式状态,此时$(>>)$函数将变得很有用。

该函数常常用在求值为类型$()$的计算中。单元类型$()$常用于求值结果没有意义的情况。和$\bot$不同的是,$()$类型
只有一个值,也就是$()$。该单元类型类似于C语言中的$void$类型,只是C的$void$类型没有取值。

$fail$函数接受一个出错字符串作为参数,返回一个代表失败的计算,而且该计算中很有可能包含给定出错字符串。
$fail$函数的默认定义是返回一个$\bot$计算,因此调用该函数将会终止程序。后面会讲到f该函数的用处,我们现在
可以先忽略它。

\subsection{Monadic函数以及Monadic计算的解释}
$return$和我们的$isqrt$这样一类函数返回的计算,或者通过Monad构造函数如$Just$得到的计算,同普通值在本质上
有些许差别。前面我曾说,可以将$monadic\ value$看成是$computation$而非普通$value$,其背景如下文。

考虑$Maybe$,一个monadic值$Just 3$并不立马等同于普通值$3$。它是一个返回$3$的计算。一眼看起来这有点不
自然,毕竟我们已经得到了值$3$,还是先试着解释绑定子函数$(>>=)$的行为吧。该函数接受一个源计算,和一个
消费者monadic函数,并将源计算的结果绑定为消费者函数的参数,而\textbf{绑定}的意义则由具体Monad解释。
由此可以得出一个内在思想:在一个计算中使用另一个计算的结果,中间不需要某个记号去运行该计算。

重要的是,只要你不请求该结果,中间就不会有结果产生,存在的只是计算本身。我们如何请求一个计算的结果呢?
最明显的方法是使用相应的monad中的绑定子函数。但你刚刚知道,绑定子函数并不产生值,它只给出一个计算,
从中取出结果并使用之只是该计算的一部分。我们如何从一个计算中取出本身并非计算的的真正最终结果呢?
换句话说,怎样运行一个计算呢?

\subsection{陷入一个Monad}
你已经看见,monad是表示计算的容器。你可以创建求得某个值的计算,可以绑定该计算结果给monadic函数。但我们
也想得到最终值,比如我们想从$Just 3$中取出$3$。至今为止,我们接触到的monad都是有构造函数的,比如$Maybe$
和列表,所以你可以通过模式匹配甚至等式从一个monadic值中解出最终值。

即使那些构造函数($Just$,$Nothing$,$[\ ]$,$(:)$)是未知的,也会有一些辅助函数用于从计算中求得结果,视不同的
Monad这些函数看起来也许会有些差别。比方说,对于列表,我们有$head$和$last$函数。而对于$Maybe$,则有函数
$fromJust$和$fromMaybe$。

但万一你既不知道这些构造函数,而且也没有辅助函数可用呢?此时确实没有任何办法可以从一个计算中结出结果。
这一点很重要,因为它使得我们可以用monad做一些有趣的事情。该属性是不用到处传递世界的状态就可以与外部世界
交互的基础 -- 它也正是$IO$ monad的基础,我们很快就会聊到它。
\clearpage
\section{隐式状态}
关于状态,在第二节,我已经谈了许多。状态在多数其他语言中相当基础且不可或缺,以至于你可能从未困扰于意识到它的
存在。通常,命令式模型的程序设计将整个程序看做一个被持续修改的巨大的状态。这没啥可惊讶的,因为多数语言都是被
建模成反映图灵机计算模型。Haskell则(且必须是)另辟蹊径。

那首先什么是状态?它是你的程序或者程序中的某个部分可以在其生命期中可以访问并修改的一个环境。要将其纳入纯
函数式编程模型并保持引用透明,状态必须是显式的,比如将其作为一个函数参数。我们再次看看第一节中的$random$
函数式如何工作的:

\begin{lstlisting}
random :: RandomState -> (Int, RandomState)
\end{lstlisting}

在非纯(non-pure)语言中,一个返回随机数的函数可以避免到处传递状态,比如,通过使用全局变量。Haskell中情况则不是
这样。当然,这是一份Monad教程,假如Monad不能帮助解决该问题,我也不会提到状态。:-)

确实,Monad真的可以搞定它。事实上,对于所有和状态相关的事物,monad是个美德。\clearpage

\section{IO Monad}
\section{语法糖}
\section{更多语法糖}
\section{回溯跟踪Monad}
\section{Monad库函数}
\section{Monad变换器}

\end{document}
