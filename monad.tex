% !TEX TS-program = xelatex
% !TEX encoding = UTF-8

% This is a simple template for a XeLaTeX document using the "article" class,
% with the fontspec package to easily select fonts.

\documentclass[11pt]{article} % use larger type; default would be 10pt

\usepackage{fontspec} % Font selection for XeLaTeX; see fontspec.pdf for documentation
\defaultfontfeatures{Mapping=tex-text} % to support TeX conventions like ``---''
\usepackage{xunicode} % Unicode support for LaTeX character names (accents, European chars, etc)
\usepackage{xltxtra} % Extra customizations for XeLaTeX

\usepackage[colorlinks=true,pdfstartview=FitH,bookmarks=true]{hyperref}

% other LaTeX packages.....
\usepackage{geometry} % See geometry.pdf to learn the layout options. There are lots.
\geometry{a4paper} % or letterpaper (US) or a5paper or....
%\usepackage[parfill]{parskip} % Activate to begin paragraphs with an empty line rather than an indent

\usepackage{listings}
\lstset{language=Haskell}
\usepackage{zhspacing}

\title{Undertanding Haskell Monads\\
深入理解 Haskell 单子}

\author{Author: Ertugrul S\"oylemez\\
\url{http://ertes.de/articles/monads.html}\\[2mm]
翻译:李群}

\date{\today\\
Version 1.03}

\zhspacing
\begin{document}
\maketitle

Haskell 是一门流行的现代函数式程序设计语言,它易于学习,语法优美,并且相当富有生产力。可是学习Haskell的过程中有一个最大的障碍,那便是单子(Monad)。虽然它们其实相当简单,但初学者很容易对其感到困惑。 我对于将Haskell推广到现实应用深怀兴趣,因此写下这篇介绍。

\renewcommand\contentsname{目录}
\tableofcontents

\section{序言}

这篇教程是写给对Haskell语言已经有了基本了解的新手的,你们或许曾经尝试理解Haskell的一个关键概念,也就是单子。假如你理解起来有难度,或者确实已经理解,但还是想有更深入的洞察或者了解其背后的动机或背景,本文便是为你而作。

Haskell是一门函数式语言。这一点并无特殊之处,但其设计使它易于学习、理解,并且实践中非常有效(effect)和高效(efficient)。Haskell有一个非常特殊的特性,也就是泛化(generalization)的概念。它的意思是,相比于直接实现一个想法,你更乐意试着寻找一个更通用的概念,使得前者是其一个隐含的特例。这种方法的优点是,假若将来又找到另一个特例,你不必重新实现之,至少不必全部推到重来。

传统的程序员从未面对过泛化问题,最多只是面向对象中的抽象概念而已。程序员更喜欢具体、特化的概念,就像直白如“干活的工具“。不幸的是,这种态度仍然普遍存在。单子的概念就是一个尤其悲情的例子,虽然单子极其有用,但是Hakell新手往往容易放弃深入理解它,因为单子是个非常抽象的结构,使得在其基础之上实现的功能也处于一个难以置信的通用级别。

或许某些读者读过 Brent Yorgey 的 《抽象、直觉以及”单子教程谬误“》,书中清楚地阐明,向Haskell新手提供单子的又一种解释为何总是于事无补。环顾所有,我将上述问题视为一个额外的麻烦:虽然掌握了抽象概念后会带来诸多便利,但很多人都试图抵制它们。因此,这份教程中我的主要意图便是帮助读者击败对抽象概念的恐惧,并阐明单子的本质:它不过是组合计算的一种风格的抽象而已。

我希望这篇文章能够给予你帮助,若有任何建设性的反馈,我将非常感谢。

\section{动机}
\section{一个示例}
\section{单子}
\section{单子的属性}
\section{隐式状态}
\section{IO单子}
\section{语法糖}
\section{更多语法糖}
\section{回溯跟踪单子}
\section{单子库函数}
\section{单子变换器}

\end{document}
