% !TEX TS-program = xelatex
% !TEX encoding = UTF-8

% This is a simple template for a XeLaTeX document using the "article" class,
% with the fontspec package to easily select fonts.

\documentclass[11pt]{article} % use larger type; default would be 10pt

\usepackage{fontspec} % Font selection for XeLaTeX; see fontspec.pdf for documentation
\defaultfontfeatures{Mapping=tex-text} % to support TeX conventions like ``---''
\usepackage{xunicode} % Unicode support for LaTeX character names (accents, European chars, etc)
\usepackage{xltxtra} % Extra customizations for XeLaTeX

\usepackage[colorlinks=true,pdfstartview=FitH,bookmarks=true]{hyperref}

% other LaTeX packages.....
\usepackage{geometry} % See geometry.pdf to learn the layout options. There are lots.
\geometry{a4paper} % or letterpaper (US) or a5paper or....
%\usepackage[parfill]{parskip} % Activate to begin paragraphs with an empty line rather than an indent

\usepackage{listings}
\lstset{frame=leftline,basicstyle=\sffamily,language=Haskell}
\usepackage{zhspacing}

\title{Undertanding Haskell Monads\\
深入理解 Haskell 单子}

\author{Author: Ertugrul S\"oylemez\\
\url{http://ertes.de/articles/monads.html}\\[2mm]
翻译:李群}

\date{\today\\
Version 1.03}

\zhspacing
\begin{document}
\maketitle

Haskell 是一门流行的现代函数式程序设计语言,它易于学习,语法优美,并且相当富有生产力。可是学习Haskell的过程中有一个最大的障碍,那便是单子(Monad)。虽然它们其实相当简单,但初学者很容易对其感到困惑。 我对于将Haskell推广到现实应用深怀兴趣,因此写下这篇介绍。

\renewcommand\contentsname{目录}
\tableofcontents

\section{序言}

这篇教程是写给对Haskell语言已经有了基本了解的新手的,你们或许曾经尝试理解Haskell的一个关键概念,也就是单子。假如你理解起来有难度,或者确实已经理解,但还是想有更深入的洞察或者了解其背后的动机或背景,本文便是为你而作。

Haskell是一门函数式语言。这一点并无特殊之处,但其设计使它易于学习、理解,并且实践中非常有效(effect)和高效(efficient)。Haskell有一个非常特殊的特性,也就是泛化(generalization)的概念。它的意思是,相比于直接实现一个想法,你更乐意试着寻找一个更通用的概念,使得前者是其一个隐含的特例。这种方法的优点是,假若将来又找到另一个特例,你不必重新实现之,至少不必全部推到重来。

传统的程序员从未面对过泛化问题,最多只是面向对象中的抽象概念而已。程序员更喜欢具体、特化的概念,就像直白如“干活的工具“。不幸的是,这种态度仍然普遍存在。单子的概念就是一个尤其悲情的例子,虽然单子极其有用,但是Hakell新手往往容易放弃深入理解它,因为单子是个非常抽象的结构,使得在其基础之上实现的功能也处于一个难以置信的通用级别。

或许某些读者读过 Brent Yorgey 的 《抽象、直觉以及”单子教程谬误“》,书中清楚地阐明,向Haskell新手提供单子的又一种解释为何总是于事无补。环顾所有,我将上述问题视为一个额外的麻烦:虽然掌握了抽象概念后会带来诸多便利,但很多人都试图抵制它们。因此,这份教程中我的主要意图便是帮助读者击败对抽象概念的恐惧,并阐明单子的本质:它不过是组合计算的一种风格的抽象而已。

我希望这篇文章能够给予你帮助,若有任何建设性的反馈,我将非常感谢。
\section{Monad之由来}

Haskell是一门纯函数式编程语言。用Haskell编写的函数是引用透明(referentially transparent)的。直觉上来说,它的意思是,对于相同的给定参数,函数总是返回同样的结果。更正式一点讲,给定一函数$f$,把对$f$的调用替换成调用结果,对程序的意义没有任何影响。所以,假如$f\ 3 = foo$,你可以安全的把所有出现$f\ 3$的地方都替换为$foo$,反之亦然。“纯函数式”意味着该语言不允许副作用(side-effects),因为它会破坏引用透明。这样一来,函数的结果仅仅依赖于给定参数,也就是说,没有副作用。

Haskell中的函数和数学意义上的函数很相似,这使得对代码进行推理(reasoning)更加简单,而且相比于非引用透明的代码,编译器很多情况下可以做出更好的优化。进一步讲,求值顺序变得毫无意义。对于表达式$(x,y)$,编译器可以自由选择先对$x$还是$y$进行求值。如果其中一个值并不需要,甚至可以忽略对其求值。这带来了灵活性(如果仅仅是使用其有限部分,你可以拥有无穷的数据结构或者计算\footnote{译者注:比如Scheme中的Lazy List})和高性能。最后,编译器还可以选择获得结果的任意执行路径,从而导致Haskell程序具有近乎疯狂的可并行性,因为编译器可以同时并行执行多条路径。比如,它可以决定同时计算出$x$和$y$。

与引用透明相反的是引用模糊(referentially opaque)。一个引用模糊的函数有多重含义,甚至对于相同参数会返回不同结果,而标准示例便是随机数生成器。在大多数程序语言中,随机数函数根本就不需要任何参数。对于一个仅仅向屏幕打印出固定文本,并且总是返回$0$的函数来讲,虽然有点违反直觉,它也是个引用模糊的函数。因为你无法在不改变程序意义的条件下,把对该函数的调用全部替换成$0$。

正如上文指出,一个明显的结果是Haskell中无法写出一个不带参数的返回伪随机数的$random$函数,因为这会破坏引用透明。事实上,Haskell中一个不带参数的函数根本就不是函数,它仅仅是个值。这个问题存在很多简单的解决方案,其中一个方法是在输入参数中引入一个状态值,函数返回伪随机数的同时也返回一个新的状态。

\begin{lstlisting}
random :: RandomState -> (Int, RandomState)
\end{lstlisting}

另一个方法是用一个参数作为初始种子值,然后返回一个包含伪随机数的无穷表。利用上面定义的$random$,该函数可以简单的实现为:

\begin{lstlisting}
randomList :: RandomState -> [Int]
randomList state = x : randomList newState
  where
    (x, newState) = random state
\end{lstlisting}

由上述示例我们得出,确定性序列的的问题可以轻松搞定,而且相比于命令式语言中常见的引用模糊的$random$函数,我们得到一个有用的特性:状态可以序列号,并且简单地回溯到先前的状态,或者向两个函数注入相同的伪随机数序列。

如何处理输入输出呢?一门通用语言如果不能开发用户接口或者读文件几乎毫无用处。我们终究想要从键盘读取输入或者向终端打印一些东西。假想遇到一个$getChar$函数,它从终端读取单个字符:

\begin{lstlisting}
getChar :: Char
\end{lstlisting}

你会发现该函数违背了引用透明,因为每次调用该函数都可能返回一个不同的字符。前面已经看到,该问题可以通过引入一个状态变量解决。但我们需要什么状态?终端的状态吗?嗯,让我们使之更通用一点,传入宇宙的状态,假设它类型是$Universe$。这样我们可以修改$getChar$函数的类型,并且实现一个$twoChars$函数以展现如何使用$getChar$:

\begin{lstlisting}
getChar :: Universe -> (Char, Universe)

twoChars :: Universe -> (Char, Char, Universe)
twoChars world0 = (c1, c2, world2)
  where
    (c1, world1) = getChar world0
    (c2, world2) = getChar world1
\end{lstlisting}

我们似乎已经找到一种有效的解决方案来应付这种问题 -- 只要传递一个状态变量即可。但该方法有个问题。首先,当然,因为需要额外传递状态,程序员需要付出更多的击键;其次,更重要的是,像随机函数$random$这样的有用而必要的功能,却成为读取键盘和写终端之类严格非纯(strictly impure)操作的主要障碍:

\begin{lstlisting}
strangeChars :: Universe -> (Char, Char)
strangeChars world = (c1, c2)
  where
    (c1, _) = getChar world
    (c2, _) = getChar world
\end{lstlisting}

让我们试着理解以上代码的意图究竟是什么。我们从宇宙中,也就是世界的状态, 读取字符$c1$,同样的状态里还读取了字符$c2$,于是我们其实时光旅行到了过去。但这是何时发生的呢?首先,计算$c1$和$c2$的顺序是未定义的,因为我们没有像在$twoChars$函数里那样序列化世界的状态。 其次,$strangeChars$没有返回更新后的宇宙状态,当数据读出之后,这事儿被忘记得一干二净\footnote{译者注:因为世界的状态没有更新},就像压根儿没发生过一样。

结论:我们可以无痛地序列化伪随机数生成器的状态,对于宇宙的状态却无能为力\footnote{译者注:伪随机数生成器可以通过公式(如线性同余算法)由上一个数值计算出下一个值,而对于宇宙状态来说,是没有公式可以描述的。}。该问题存在一些解决方案。比如,纯函数式语言$Clean$,它使用了一个$uniqueness$类型,基本上就像上文描述的$universe$那样。但是,该语言会检测并阻止任何对世界状态的并行访问的尝试,从而保证了其基于显式状态传递的I/O一致性。Haskell则采用了另一种方案。它引入了一种范畴论中称之为``Monad''的结构,而不是显式地传递世界的状态。

\section{一个示例}
\section{单子}
\section{单子的属性}
\section{隐式状态}
\section{IO单子}
\section{语法糖}
\section{更多语法糖}
\section{回溯跟踪单子}
\section{单子库函数}
\section{单子变换器}

\end{document}
