\section{序言}

这篇教程是写给对Haskell语言已经有了基本了解的新手的,你们或许曾经尝试理解Haskell的一个关键概念,也就是单子。假如你理解起来有难度,或者确实已经理解,但还是想有更深入的洞察或者了解其背后的动机或背景,本文便是为你而作。

Haskell是一门函数式语言。这一点并无特殊之处,但其设计使它易于学习、理解,并且实践中非常有效(effect)和高效(efficient)。Haskell有一个非常特殊的特性,也就是泛化(generalization)的概念。它的意思是,相比于直接实现一个想法,你更乐意试着寻找一个更通用的概念,使得前者是其一个隐含的特例。这种方法的优点是,假若将来又找到另一个特例,你不必重新实现之,至少不必全部推到重来。

传统的程序员从未面对过泛化问题,最多只是面向对象中的抽象概念而已。程序员更喜欢具体、特化的概念,就像直白如“干活的工具“。不幸的是,这种态度仍然普遍存在。单子的概念就是一个尤其悲情的例子,虽然单子极其有用,但是Hakell新手往往容易放弃深入理解它,因为单子是个非常抽象的结构,使得在其基础之上实现的功能也处于一个难以置信的通用级别。

或许某些读者读过 Brent Yorgey 的 《抽象、直觉以及”单子教程谬误“》,书中清楚地阐明,向Haskell新手提供单子的又一种解释为何总是于事无补。环顾所有,我将上述问题视为一个额外的麻烦:虽然掌握了抽象概念后会带来诸多便利,但很多人都试图抵制它们。因此,这份教程中我的主要意图便是帮助读者击败对抽象概念的恐惧,并阐明单子的本质:它不过是组合计算的一种风格的抽象而已。

我希望这篇文章能够给予你帮助,若有任何建设性的反馈,我将非常感谢。